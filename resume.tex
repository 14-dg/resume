\documentclass[11pt,a4paper]{article}

\usepackage[margin=2cm]{geometry}
\usepackage{enumitem}
\usepackage{titlesec}
\usepackage{parskip}
\usepackage[T1]{fontenc}
\usepackage[utf8]{inputenc}
\usepackage{lmodern}
\usepackage[hidelinks]{hyperref}

\titleformat{\section}{\large\bfseries}{}{0em}{}[\titlerule]

\begin{document}

%====================
% PERSÖNLICHE DATEN
%====================

\begin{center}
{\LARGE \textbf{Daniel Gräf}}\\
\vspace{0.2cm}
Zedernweg 4, 41569 Rommerskirchen \\
+49 1523 1094514 \\
daniel.graef14@gmail.com
\end{center}

%====================
\section*{Persönliches Profil}
%====================

Technisch analytischer Student der Technischen Informatik (5. Semester) mit
Schwerpunkt auf systemnaher Softwareentwicklung, Netzwerktechnik und
sensorbasierter Systemintegration. Industrieerfahrung im Rheinmetall-Konzern
(Pierburg) sowie eigenständig realisierte Hard- und Softwareprojekte mit
mehreren tausend Zeilen Code. Hohe Eigenverantwortung, strukturierte
Arbeitsweise und ausgeprägtes Systemverständnis.

%====================
\section*{Ausbildung}
%====================

\textbf{Technische Hochschule Köln} \hfill Seit WS 2023 \\
B.Sc. Technische Informatik (aktuell 5. Semester)

\begin{itemize}[leftmargin=*]
    \item Module u.a.: Netze und Protokolle (CCNA-nah), C-Programmierung,
    Java, Rechnerarchitektur, Digitaltechnik
    \item Schwerpunkt: Systemnahe Programmierung und technische Softwareentwicklung
\end{itemize}

\textbf{Pascal Gymnasium Grevenbroich} \hfill Abitur 2023 (1,6) \\
Verkürzte Schulzeit (Überspringen der 9. Klasse)

%====================
\section*{Berufserfahrung}
%====================

\textbf{Werkstudent} \hfill Pierburg GmbH (Rheinmetall-Konzern)

\begin{itemize}[leftmargin=*]
    \item Entwicklung interner Softwarelösungen zur Prozessautomatisierung
    \item Konzeption und Implementierung eines Python-basierten Analyse-Tools
    \item Verarbeitung und strukturierte Analyse technischer Dokumente
    \item Produktiver Einsatz im industriellen Umfeld
\end{itemize}

%====================
\section*{Technische Kompetenzen}
%====================

\textbf{Programmiersprachen}
\begin{itemize}[leftmargin=*]
    \item C (Speicherverwaltung, Pointer, strukturierte Programmierung)
    \item Python (Automatisierung, Datenverarbeitung, Systemintegration)
    \item Java (OOP, Softwarearchitektur)
\end{itemize}

\textbf{System- und Hardwarekenntnisse}
\begin{itemize}[leftmargin=*]
    \item Embedded-nahe Entwicklung (Arduino, Raspberry Pi)
    \item Sensorintegration (LiDAR, Ultraschall, Gyroskop)
    \item Grundlagen Mikrocontroller-Programmierung
\end{itemize}

\textbf{Netzwerk- und Systemwissen}
\begin{itemize}[leftmargin=*]
    \item TCP/IP, Routing, Subnetting (CCNA-Niveau)
    \item Linux-Umgebung, Shell, Git
\end{itemize}

%====================
\section*{Technische Projekte}
%====================

\textbf{Autonomer Rover (Eigenentwicklung)}
\begin{itemize}[leftmargin=*]
    \item Konstruktion eines 3-Achsen-Rovers mit 6 Rädern und Panzerlenkung
    \item Integration mehrerer Sensorsysteme (LiDAR, Ultraschall, Gyrosensorik)
    \item Entwicklung der Steuerungs- und Auswertelogik
    \item Ziel: Autonome Navigation zwischen GPS-Punkten mit Hinderniserkennung
\end{itemize}

\textbf{PDF-Analyse-Tool (Industrieeinsatz)}
\begin{itemize}[leftmargin=*]
    \item >3000 Zeilen Python-Code
    \item Automatisierte Analyse großer Dokumentenmengen
    \item Optimierung interner Arbeitsprozesse
\end{itemize}

%====================
\section*{Berufliches Ziel}
%====================

Mitarbeit an technisch anspruchsvollen Entwicklungsprojekten im Bereich
robuster, sicherheitskritischer oder eingebetteter Systeme mit langfristiger
Perspektive in Forschung und Entwicklung.

\end{document}