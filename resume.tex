\documentclass[11pt,a4paper]{article}

\usepackage[margin=2cm]{geometry}
\usepackage{parskip}
\usepackage{titlesec}
\usepackage{enumitem}
\usepackage[T1]{fontenc}
\usepackage[utf8]{inputenc}
\usepackage{lmodern}
\usepackage[hidelinks]{hyperref}

\titleformat{\section}{\large\bfseries}{}{0em}{}[\titlerule]

\begin{document}

\begin{center}
{\LARGE \textbf{Daniel Gräf}}\\
Technische Informatik (B.Sc.) \\
Rommerskirchen, Deutschland \\
+49 1523 1094514 \quad | \quad daniel.graef14@gmail.com
\end{center}

%================================================
\section*{Berufserfahrung}
%================================================

\textbf{Werkstudent – Pierburg GmbH (Rheinmetall-Konzern)} \hfill Seit 2024

\begin{itemize}[leftmargin=*]
    \item Entwicklung Python-basierter Automatisierungslösungen im technischen Umfeld
    \item Konzeption und Implementierung mehrerer produktiv eingesetzter Analyse-Tools
    \item Automatisierte Verarbeitung technischer Dokumentationen (INEOS)
    \item Signifikante Reduktion manueller Analyse- und Suchaufwände
\end{itemize}

%================================================
\section*{Ausbildung}
%================================================

\textbf{Technische Hochschule Köln} \hfill 2023 – 2026 \\
Bachelor Technische Informatik \\
Notenschnitt: 1,7 \\
Abschluss: Sommer 2026 (verkürzte Studienzeit, 6 Semester)

\begin{itemize}[leftmargin=*]
    \item Vollständiger Abschluss aller Bachelor-Module
    \item Module u.a.: Softwareentwicklung (Java, C), Rechnerarchitektur,
    Netzwerktechnik (CCNA-nah), Digitaltechnik
    \item Geplanter Master: Technische Informatik (3 Semester) ab WS 2026
\end{itemize}

%================================================
\section*{Technische Kompetenzen}
%================================================

\textbf{Programmiersprachen}
\begin{itemize}[leftmargin=*]
    \item Python (fortgeschritten): NumPy, Pandas, TensorFlow
    \item Java (sehr gute Kenntnisse, OOP, Anwendungsarchitektur)
    \item C (Grundlagen systemnaher Programmierung)
\end{itemize}

\textbf{Algorithmen \& Systementwicklung}
\begin{itemize}[leftmargin=*]
    \item Implementierung und Visualisierung von A*, Dijkstra
    \item Entwicklung von Simulations- und Validierungstools
    \item Asynchrone Programmierung und strukturierte Softwarearchitektur
\end{itemize}

\textbf{Systemintegration}
\begin{itemize}[leftmargin=*]
    \item Sensorintegration (LiDAR, Ultraschall, Gyroskop)
    \item KNX/EIB-Integration über HomeAssistant
    \item Linux-Umgebung, Git
\end{itemize}

%================================================
\section*{Ausgewählte Projekte}
%================================================

\textbf{PDF-Automatisierungssysteme (Industrieeinsatz)}
\begin{itemize}[leftmargin=*]
    \item Entwicklung mehrerer eigenständig konzipierter Tools zur 
    strukturierten Analyse technischer PDF-Dokumente
    \item Unternehmensweite Nutzung zur Effizienzsteigerung
    \item Vollständige Architektur, Implementierung und Wartung in Eigenverantwortung
\end{itemize}

\textbf{Autonomer Rover – Systemkonzept und Simulation}
\begin{itemize}[leftmargin=*]
    \item Vollständige Hardware- und Softwareplanung eines sensorbasierten Systems
    \item Entwicklung eines Pathfinder-Simulators zur Validierung von Navigationsalgorithmen
    \item Fokus auf autonome Navigation und Hinderniserkennung
\end{itemize}

\textbf{Systementwurfspraktikum – Teamleitung}
\begin{itemize}[leftmargin=*]
    \item Leitung eines 5-köpfigen Teams
    \item Entwicklung einer Raumfindungs-App für die TH Köln
    \item Architekturdefinition, Aufgabenverteilung und technische Umsetzung
\end{itemize}

%================================================
\section*{Interessen}
%================================================

Automatisierung technischer Prozesse, KI-gestützte Datenanalyse,
Simulation komplexer Systeme, robuste Softwarearchitekturen.

\end{document}