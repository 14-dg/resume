\documentclass[11pt,a4paper]{article}

\usepackage[margin=2cm]{geometry}
\usepackage{parskip}
\usepackage{titlesec}
\usepackage{enumitem}
\usepackage[T1]{fontenc}
\usepackage[utf8]{inputenc}
\usepackage{lmodern}
\usepackage[hidelinks]{hyperref}

\titleformat{\section}{\large\bfseries}{}{0em}{}[\titlerule]

\begin{document}

\begin{center}
{\LARGE \textbf{Daniel Gräf}}\\
B.Sc. Technische Informatik (Abschluss Sommer 2026) \\
Rommerskirchen, Deutschland \\
+49 1523 1094514 \quad | \quad daniel.graef14@gmail.com
\end{center}

%================================================
\section*{Profil}
%================================================

Softwareentwickler mit starkem Fokus auf Python-basierter Systementwicklung,
Softwarearchitektur und Automatisierung technischer Prozesse.
Erfahrung in der eigenständigen Konzeption, Umsetzung und Wartung
komplexer Anwendungen mit industriellem Mehrwert.
Sehr gutes Systemverständnis, analytische Arbeitsweise und
hohe technische Eigenverantwortung.

%================================================
\section*{Berufserfahrung}
%================================================

\textbf{Werkstudent – Pierburg GmbH (Rheinmetall-Konzern)} \hfill Seit 2024

\begin{itemize}[leftmargin=*]
    \item Entwicklung webbasierter Oberflächen mit \textit{optiSLang} und \textit{pyOwa}
    \item Automatisierung komplexer CAD-Simulations-Workflows
    \item Abbildung technischer Parameter und Simulationsergebnisse
          in benutzerfreundlichen Webinterfaces
    \item Reduktion der Abhängigkeit von Simulationsexperten
          durch strukturierte Parametrisierung und Ergebnisaufbereitung
\end{itemize}

%================================================
\section*{Ausbildung}
%================================================

\textbf{Technische Hochschule Köln} \hfill 2023 – 2026 \\
Bachelor Technische Informatik \\
Notenschnitt: 1,7 \\
Abschluss in verkürzter Studienzeit (6 Semester)

\begin{itemize}[leftmargin=*]
    \item Module u.a.: Softwareentwicklung (Java, C),
          Rechnerarchitektur, Netzwerktechnik, Digitaltechnik
    \item Geplanter Master: Technische Informatik (ab WS 2026)
\end{itemize}

%================================================
\section*{Technische Kompetenzen}
%================================================

\textbf{Programmiersprachen}
\begin{itemize}[leftmargin=*]
    \item Python (sehr erfahren)
    \item Java (sehr gute Kenntnisse)
    \item Kotlin (fortgeschritten)
    \item SQL (fortgeschritten)
    \item C (Grundlagen systemnaher Programmierung)
\end{itemize}

\textbf{Frameworks \& Technologien}
\begin{itemize}[leftmargin=*]
    \item React (fortgeschritten)
    \item Docker (praktische Erfahrung)
    \item Kubernetes (Grundverständnis der Prinzipien)
    \item Linux (sehr sicherer Umgang)
\end{itemize}

\textbf{Software Engineering}
\begin{itemize}[leftmargin=*]
    \item Softwarearchitektur und modulare Systemgestaltung
    \item Asynchrone Programmierung
    \item Implementierung und Visualisierung von Algorithmen (A*, Dijkstra)
    \item NumPy, Pandas, TensorFlow
\end{itemize}

%================================================
\section*{Ausgewählte Projekte}
%================================================

\textbf{PDF-Automatisierungstools (Industrieeinsatz bei INEOS)}
\begin{itemize}[leftmargin=*]
    \item Entwicklung mehrerer Python-Tools zur strukturierten Analyse
          technischer PDF-Dokumente
    \item Produktiver Einsatz zur signifikanten Reduktion manueller Recherchearbeit
    \item Eigenständige Architektur, Implementierung und Weiterentwicklung
\end{itemize}

\textbf{Autonomes Rover-System}
\begin{itemize}[leftmargin=*]
    \item Vollständige Hard- und Softwareplanung eines sensorbasierten Systems
    \item Entwicklung eines Simulations- und Visualisierungstools
          zur Validierung von Navigationsalgorithmen
    \item Integration von LiDAR-, Ultraschall- und Gyrosensorik
\end{itemize}

\textbf{Systementwurfspraktikum – Teamleitung}
\begin{itemize}[leftmargin=*]
    \item Leitung eines 5-köpfigen Entwicklerteams
    \item Konzeption und Umsetzung einer Raumfindungs-App für die TH Köln
    \item Architekturdefinition, Aufgabenstrukturierung und Implementierung
\end{itemize}

%================================================
\section*{Weitere technische Projekte}
%================================================

\textbf{HomeAssistant-System (Eigenentwicklung)}
\begin{itemize}[leftmargin=*]
    \item Aufbau und Integration eines vollautomatisierten Smart-Home-Systems
    \item KNX/EIB-Anbindung zur Steuerung von Temperatur, Fensterstatus
          und Warmwasser
    \item Integration von Kalender- und Ereignislogik zur Alltagsautomatisierung
\end{itemize}

\end{document}